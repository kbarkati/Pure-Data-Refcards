\documentclass[a4paper, landscape, 10pt]{article}
\usepackage[frenchb]{babel}
\usepackage[utf8]{inputenc}
\usepackage[T1]{fontenc}
\usepackage{xspace}
\usepackage{graphicx}
\usepackage{url}
\usepackage{array}
\usepackage{listings}
\usepackage{multicol}
\usepackage{tabularx}
\usepackage{geometry}
\usepackage[usenames]{color}
\usepackage{hyperref}
\hypersetup{%
  a4paper = true,
  pdfstartview = FitH,
  colorlinks = true,
  linkcolor = black,
  citecolor = black,
  urlcolor = black, % blue,
  bookmarksopen = true,
  bookmarksopenlevel = 2,
  bookmarks = true
}

\urlstyle{sf}
\thispagestyle{empty}
\geometry{hmargin=0.8cm, vmargin=0.7cm}
% \renewcommand{\columnseprule}{0.4pt}
\newcommand{\puredata}{Pure Data\xspace}
\newcommand{\email}[1]{\href{mailto:#1}{\textsf{#1}}}
\newcommand{\refcardtitle}[1]{
  \begin{center}
    \textbf{\small{#1}} %\\
  \end{center}
}


\begin{document}
\begin{multicols}{3}

  \begin{center}
    \Large{\textbf{Carte de référence de \puredata}} \\
    \small{Karim \textsc{Barkati} -- \today} 
  \end{center}

  \footnotesize

  \refcardtitle{Modes}
  \begin{tabularx}{9.1cm}{X}
    \texttt{ctl-e} (ou \texttt{cmd-e}) alterne entre le mode \emph{jeu} (performance) et le mode \emph{édition} (programmation); cela modifie l'action des clics de la souris sur le patch.% (performance vs programming). %Use \emph{run} mode for performance, \emph{edit} mode to change the patch.
  \end{tabularx}
  \refcardtitle{Colle}
  \begin{tabularx}{9cm}{>{\tt}l X}
    % \multicolumn{2}{c}{Glue} \\
    bang & retourne un message \emph{bang} \\
    float & stocke et rappelle un nombre \\
    symbol & stocke et rappelle un symbole \\
    int & stocke et rappelle un entier \\
    send & envoie un message à un objet nommé \\
    receive & reçoit les messages envoyés par \texttt{send} \\
    select & compare des nombres et/ou des symboles \\
    route & oriente les messages selon le premier élément \\
    pack & combine plusieurs atomes en un seul message \\
    unpack & décompose un message en atomes séparés \\
    trigger & déclenche en séquence et convertit des messages \\
    spigot & (robinet) ouvre et ferme le passage de messages \\
    moses & (moïse) sépare un flux de nombres en deux sorties \\
    until & mécanisme de bouclage \\
    print & imprime des messages sur la console \\
    makefilename & formate un symbole comportant une variable \\
    change & filtre les répétitions dans un flux de nombres \\
    swap & permute deux nombres \\
    value & valeur numérique partagée (variable globale) \\
  \end{tabularx}


  \refcardtitle{Temps}
  \begin{tabularx}{9cm}{>{\tt}l X}
    delay & envoie un message après un délai \\
    metro & envoie un message périodiquement \\
    line & envoie une suite linéraire de nombres \\
    timer & mesure des intervalles temporels \\
    cputime & mesure le temps CPU \\
    realtime & mesure le temps par le système d'exploitation \\
    pipe & ligne à retard (extensible) pour les messages \\
  \end{tabularx}

  \refcardtitle{Maths}
  \begin{tabularx}{9cm}{>{\tt}X X}
    + - * / pow  & arithmétique \\
    == != > < >= <= & tests de comparaison \\
    \& \&\& | || \% & opérations logiques ou bit-à-bit \\
    mtof ftom powtodb rmstodb dbtopow dbtorms & convertions acoustiques \\
    mod div sin cos tan atan atan2 sqrt log exp abs & maths supérieures \\
    random expr & maths inférieures \\ 
    max min & le plus grand ou le plus petit \\
    clip & contraint un nombre à rester dans un intervalle borné \\
  \end{tabularx}

  % \columnbreak

  \refcardtitle{Midi}
  \begin{tabularx}{9cm}{>{\tt}X l}
    notein ctlin pgmin bendin touchin polytouchin midiin sysexin & entrées MIDI \\
    noteout ctlout pgmout bendout touchout polytouchout midiout & sorties MIDI \\
  \end{tabularx}
  \begin{tabularx}{9cm}{>{\tt}l X}
    % makenote & schedule a delayed "note off" message corresponding to a note-on \\ 
    makenote & envoie les \emph{note-on} et fabrique les \emph{note-off} à retarder \\
    % stripnote & strip "note off" messages \\
    stripnote & supprime les messages \emph{note-off} de l'entrée \\
  \end{tabularx}

  \columnbreak
  \refcardtitle{Tables}
  \begin{tabularx}{9cm}{>{\tt}l X}
    tabread & lit un nombre dans une table \\
    % tabread4 & read a number from a table, with 4 point interpolation \\
    tabread4 & lit dans une table avec une interpolation à 4 points \\
    tabwrite & écrit un nombre dans une table \\
    soundfiler & lit et écrit des tables depuis/vers des fichiers audio \\
  \end{tabularx}


  \refcardtitle{Divers}
  \begin{tabularx}{9cm}{>{\tt}l X}
    loadbang & émet un \emph{bang} au démarrage \\
    serial & contrôleur série, pour NT seulement \\
    netsend & envoie des messages sur internet \\
    netreceive & reçoit les messages de \texttt{netsend} \\
    qlist & séquenceur de messages depuis un fichier texte \\
    textfile & convertit des fichiers en messages \\
    openpanel & fenêtre \og Ouvrir\fg \\
    savepanel & fenêtre \og Enregistrer sous...\fg \\
    bag & ensemble de nombres \\
    poly & allocation polyphonique de voies \\
    key, keyup & valeurs numériques des touches du clavier \\
    keyname & nom symbolique des touches du clavier \\
  \end{tabularx}

  \refcardtitle{Maths audio}
  \begin{tabularx}{9cm}{>{\tt}l X}
    +\~\ -\~\ *\~\ /\~\ & arithmétique sur les signaux audio \\
    max\~\ min\~\ & maximum et minimum de 2 entrées audio \\
    clip\~ & contraint un signal entre deux bornes \\
    q8\_rsqrt\~\ & racine carrée inverse rapide (attention 8 bits!) \\
    q8\_sqrt\~\ & racine carrée rapide (attention 8 bits!) \\
    wrap\~\ & reste modulo 1 (partie décimale pour les positifs) \\
    fft\~\ & transformée de Fourier discrète complexe \\
    ifft\~\ & transformée de Fourier discrète inverse complexe \\
    rfft\~\ & transformée de Fourier discrète réelle \\
    rifft\~\ & transformée de Fourier discrète inverse réelle \\
    framp\~ & estimation de la fréquence et de l'amplitude FFT \\
  \end{tabularx}
  \begin{tabularx}{9cm}{>{\tt}X l}
    mtof\~\ ftom\~\ rmstodb\~\ dbtorms\~\ rmstopow\~\ powtorms\~ & conversions acoustiques \\
  \end{tabularx}

  \refcardtitle{Colle audio}
  \begin{tabularx}{9cm}{>{\tt}l X}
    dac\~\ & sortie audio \\
    adc\~\ & entrée audio \\ 
    sig\~\ & convertit les nombres en signal audio \\ 
    line\~\ & génère des rampes audio \\ 
    vline\~\ & génère des rampes audio haute-précision \\ 
    threshold\~\ & détecte le franchissement d'un seuil par un signal \\
    snapshot\~\ & échantillonne un signal (le convertit en nombre) \\
    vsnapshot\~\ & échantillonne un signal en haute-précision \\
    bang\~\ & envoie un message \emph{bang} après chaque block DSP \\
    samplerate\~\ & récupère le taux d'échantillonnage \\
    send\~\ & connexions audio à distance \og one-to-many\fg \\
    receive\~\ & reçoit le signal du \texttt{send\~} du même nom \\
    throw\~\ & envoie à distance dans un bus additionneur \\
    catch\~\ & définit et lit dans un bus additionneur \\
    block\~\ & spécifie la taille de bloc et le chevauchement \\
    switch\~\ & démarre et stoppe le calcul DSP \\
    readsf\~\ & lit un fichier audio depuis le disque dur \\
    writesf\~\ & enregistre un fichier audio sur le disque dur \\
  \end{tabularx}

  \medskip{} 
  \noindent{}
  \begin{tabularx}{9cm}{X}
%    \tiny{Copyright \copyright{ 2010} Karim \textsc{Barkati} <\email{karim.barkati@gmail.com}>, under the GNU Free Documentation License: permission is granted to make and distribute copies of this card provided the copyright notice and this permission notice are preserved on all copies.}
    \tiny{Copyright \copyright{ 2010} Karim \textsc{Barkati} <\email{karim.barkati@gmail.com}>, Permission est accordée de copier, distribuer et/ou modifier ce document selon les termes de la Licence de Documentation Libre GNU GFDL version 1.3 ou toute version ultérieure publiée par la Free Software Foundation; sans Sections Invariables; sans Textes de Première de Couverture, et sans Textes de Quatrième de Couverture.}
  \end{tabularx}
  \footnotesize

  \columnbreak
  \refcardtitle{Oscillateurs et tables audio}
  \begin{tabularx}{9cm}{>{\tt}l X}
    phasor\~\ & générateur d'ondes en dents de scie \\
    cos\~\ & cosinus \\
    osc\~\ & oscillateur cosinusoïdal \\
    tabwrite\~\ & écrit dans une table \\
    tabplay\~\ & rejoue une table (sans transposition) \\
    tabread\~\ & lit une table (sans interpolation) \\
    tabread4\~\ & lit une table avec interpolation à 4 points \\
    tabosc4\~\ & oscillateur de table d'onde avec interpolation \\ 
    tabsend\~\ & écrit continuement un bloc dans une table \\
    tabreceive\~\ & lit continuement un bloc dans une table \\
  \end{tabularx}

  \refcardtitle{Filtres audio}
  \begin{tabularx}{9cm}{>{\tt}l X}
    vcf\~\ & filtre passe-bande contrôlé par voltage \\
    noise\~\ &  générateur de bruit blanc \\
    env\~\ & suiveur d'enveloppe (amplitude RMS en dB) \\
    hip\~\ &  filtre passe-haut \\
    lop\~\ &  filtre passe-bas \\ 
    bp\~\ &  filtre passe-bande \\
    biquad\~\ & filtre brut (2 pôles et 2 zéros) \\ 
    samphold\~\ & échantillone la valeur d'un signal et la maintient \\
    print\~\ & affiche un ou plusieurs "blocs" \\
    rpole\~\ & filtre brut 1-pôle réel \\
    rzero\~\ & filtre brut 1-zéro réel \\
    rzero\_rev\~\ & filtre brut 1-zéro réel inversé en temps \\
  \end{tabularx}
  \texttt{cpole\~\, czero\~\, czero\_rev}\ \ \ \ idem en complexes 
  % \begin{tabularx}{9cm}{>{\tt}X l}
  % %   cpole\~\, czero\~\, czero\_rev & corresponding complex-valued filters \\
  %   cpole\~\, czero\~\, czero\_rev & corresponding complex filters \\
  % \end{tabularx}

  \refcardtitle{Délai audio}
  \begin{tabularx}{9cm}{>{\tt}l X}
    delwrite\~\ & écrit dans une ligne à retard \\
    delread\~\ & lit une ligne à retard \\
    vd\~\ & lit une ligne à retard avec un délai variable \\
  \end{tabularx}

  \refcardtitle{Sous-patchs}
  \begin{tabularx}{9cm}{>{\tt}l X}
    pd & définit un sous-patch \\
    table & tableau de nombres dans un sous-patch \\
    inlet & ajoute une entrée à un sous-patch \\
    outlet & ajoute une sortie à un sous-patch \\
    inlet\~\, outlet\~\ & versions audio de \texttt{inlet} et \texttt{outlet} \\
  \end{tabularx}

  \refcardtitle{Modèles de données}
  \begin{tabularx}{9cm}{>{\tt}l X}
    struct & définit une structure de données \\
    drawcurve, filledcurve & dessine une courbe \\
    drawpolygon, filledpolygon & dessine un polygone \\
    plot & trace le champ d'un tableau \\
    drawnumber & affiche une valeur numérique \\
  \end{tabularx}

  \refcardtitle{Accès aux données}
  \begin{tabularx}{9cm}{>{\tt}l X}
    pointer & pointe sur un objet appartenant à un modèle \\
    get & récupère des champs numériques \\
    set & modifie des champs numériques \\
    element & récupère un élément de tableau \\
    getsize & récupère la taille d'un tableau \\
    setsize & modifie la taille d'un tableau \\
    append & ajoute un élément à une liste \\
    % sublist & get a pointer into a list which is an element of another scalar \\
    sublist & récupère une liste depuis le champ d'un scalaire \\
%    scalar & draw a scalar on parent \\ % obsolete
  \end{tabularx}



\end{multicols}

\end{document}
